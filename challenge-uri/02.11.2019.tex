\documentclass{article}
\usepackage[utf8]{inputenc}


\begin{document}


\section{Challenge-uri}
    \begin{enumerate}
        \item Cititi un numar $n$ din terminal. Afisati, utilizand operatorul ternar si fara a folosi $and$ daca $n$ e divizibil si cu $2$ si cu $3$.
        \item Scrieti o functie care calculeaza suma: $1^{2}$ + $5^{2}$ + $9^{2}$ + ... + $n^{2}$
        \item Scrieti o functie care primeste ca parametri o lista, sa ii spunem $l$ si un numar $n$. Puneti in lista, in ordinea asta: $1^{n}$, $2^{n}$, $3^{n}$, ..., $n^{n}$, o data folosind operatorul de ridicare la putere (**) si o data fara el.
        \item Cititi un numar $n$ din terminal si generati in mod aleatoriu o lista de $n$ numere intre 1 si 1000. Afisati maximul si minimul din lista.
        \item Cititi un numar $n$ din terminal si generati in mod aleatoriu o lista de $n$ numere intre 1 si 10. Scrieti o functie care primeste lista ca parametru si returneaza diferenta dintre numarul de valori pare si cel de valori impare.
        \item Scrieti o functie care primeste ca parametru un numar $n$ si afiseaza toti divizorii lui $n$ ($i$ este divizor al lui $x$ daca $n$ mod $i$ = $0$ si $i$ $\leq$ $x$)
        \item Scrieti o functie care primeste ca parametru un numar $n$ si returneaza $"prim"$ daca $x$ este prim si $"compus"$ altfel (un numar $n$ este prim daca ii are ca divizori doar pe $1$ si pe $n$)
        \item Asta e mai grea: Scrieti un program care calculeaza in cate moduri putem aseza $n$ copii ($n$ citit din terminal) in linie dreapta daca ordinea lor conteaza. (HINT: pentru primul loc, putem face $n$ alegeri; pentru al doilea, putem alege oricare copil din cei $n$-1 inca nealesi...)
    \end{enumerate}

\end{document}
