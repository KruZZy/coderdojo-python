\documentclass{article}
\usepackage[utf8]{inputenc}
\usepackage{amsmath, amssymb, tabularx}
\begin{document}


\section{Challenge-uri}
    \begin{enumerate}
        \item Am spus data trecuta ca orice numar $n$ poate fi scris ca o suma unica de termeni ai $sirului lui Fibonacci$. Scrieti un program care face asta. HINT: ii generati pana cand dati de unul mai mare decat $n$, apoi gasiti la fiecare pas cel mai mare numar Fibonacci mai mic sau egal cu $n$, dupa care reactualizati $n$-ul.
        \item Cititi doua numere, $a$ si $n$ ($a$ $\leq$ $n$) din terminal, apoi descoperiti regula si generati lista:
            \[\begin{bmatrix}
                F[0] & ... & F[a-1] & F[a] & F[a+1] & F[a+2] & F[a+3] & F[a+4] & ... & F[n]\\
                0  & ... & 0 & 1 & 1 & 2 & 4 & 8 & ... & ?\\
              \end{bmatrix}\]
        \item Descoperiti regula si generati urmatoarea lista:
        \[\begin{bmatrix}
                S[0] & S[1] & S[2] & S[3] & S[4] & S[5] &  ... & S[n]\\
                'a' & 'b' & 'ba' & 'bab' & 'babba' & 'babbabab' & ... & ?
              \end{bmatrix}\]
        \item Cititi un numar natural $n$ din terminal si aflati suma divizorilor sai si afisati cati divizori pari si cati divizori impari are.
        \item Se citeste un numar $n$ din terminal. Aflati numarul care are maxim de divizori din intervalul $[1, n]$ (adica $1$, $2$, $3$, ... , $n-1$, $n$)

    \end{enumerate}

\end{document}
