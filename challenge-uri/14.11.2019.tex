\documentclass{article}
\usepackage[utf8]{inputenc}
\usepackage{amsmath, amssymb, tabularx}
\begin{document}


\section{Challenge-uri}
    \begin{enumerate}
        \item Cititi un numar $n$ din terminal si afisati al $n$-lea numar par si al $n$-lea numar impar.
        \item Descoperiti regula si generati lista $F$: (rezultatul e de retinut, amintiti-mi sa discutam)
            \[\begin{bmatrix}
                F[0] & F[1] & F[2] & F[3] & F[4] & F[5] & ... & F[n] \\
                0 & 1 & 1 & 2 & 3 & 5 & ... & ?\\
              \end{bmatrix}\]
        \item Verificati daca un numar $n$, citit din terminal, se afla in lista $F$ de la $1.$
        \item Folosindu-va tot de sirul de la $1.$, afisati:
            \[\dfrac{F[1]}{F[2]}, \dfrac{F[2]}{F[1]}, \dfrac{F[3]}{F[2]}, \dfrac{F[4]}{F[3]}, ..., \dfrac{F[n]}{F[n-1]}\]
            Rulati pentru diferite valori mai mari sau mai mici ale lui $n$. Ce observati?
        \item Generati aleatoriu o lista $A$ cu $n$ elemente, $n$ citit din terminal. Gasiti doua pozitii, sa le zicem $i$ si $j$, pentru care diferenta $A[i] - A[j]$ este cea maxima posibila.
        \item Cand nu ingroapa ghinde intr-un mod cel putin ciudat, $Veveritoiului$ ii place sa rezolve probleme de matematica distractiva. Sotia lui, $Veveritiana$, tocmai i-a propus urmatorul puzzle: ii da o grila cu 6 coloane notate cu $A$, $B$, $C$, $D$, $E$, $F$ si cu un numar infinit de linii. Grila va fi completata cu numere naturale, începând cu $1$. Pe liniile impare completarea se va face de la stanga la dreapta, iar pe cele pare de la dreapta la stanga. Ultimul numar de pe o linie va fi identic cu penultimul numar (in sensul completarii) de pe aceeasi linie. Pentru a intelege, i-a aratat primele 6 linii ale grilei:


        \begin{tabularx}{0.8\textwidth} {
          | >{\centering\arraybackslash}X
          | >{\centering\arraybackslash}X
          | >{\centering\arraybackslash}X
          | >{\centering\arraybackslash}X
          | >{\centering\arraybackslash}X
          | >{\centering\arraybackslash}X | }
         \hline
         A & B & C & D & E & F\\
         \hline
         1 & 2 & 3 & 4 & 5 & 5\\
        \hline
         10 & 10 & 9 & 8 & 7 & 6\\
        \hline
         11 & 12 & 13 & 14 & 15 & 15\\
         \hline
         20 & 20 & 19 & 18 & 17 & 16\\
         \hline
         21 & 22 & 23 & 24 & 25 & 25\\
         \hline
         30 & 30 & 29 & 28 & 27 & 26\\
         \hline
        \end{tabularx}


    \end{enumerate}

\end{document}
